\documentclass{article}

\usepackage[utf8]{inputenc} % Required
\usepackage{lettrine} % Make first letter large
\usepackage[base]{babel} % Proper Latin hyphens
\usepackage{fontspec} % Use custom fonts
\usepackage{lipsum} % Generate Random Text (Lorem Ipsum)
\usepackage[a4paper, total={6in, 8in}]{geometry} % Change format to A4
\usepackage{multicol} % Three Columns (SN wants that)
\usepackage{fancyhdr} % Header
\usepackage[dvipsnames]{xcolor} % Many colors (Many)
\usepackage{lmodern}
\usepackage[T1]{fontenc} 

% The color SN uses for
% the header
\definecolor{SNBlue}{HTML}{005AAF} % Whoever decided to call HEX color codes HTML Color Codes should be tried for murder

% Customizing the header
\pagestyle{fancy}
\fancyhf{}
\fancyhead[R]{\textbf{Dec. 24, 2014}}
\fancyhead[L]{\textbf{Page 4}}
\fancyhead[C]{\textbf{\color{SNBlue}{SPACEPORT NEWS}}}

% Thicccer header
\renewcommand{\headrulewidth}{1pt}
% Set Header Color
\renewcommand{\headrule}{{%
    \color{SNBlue}\hrule height \headrulewidth\hfill}
}

\title{MarsAnalysis}
\date{December 2021}

% Font
\setmainfont[
    Path=fonts/,
    BoldItalicFont=Carlito-BoldItalic.ttf,
    BoldFont=Carlito-Bold.ttf,
    ItalicFont=Carlito-Italic.ttf,
    SizeFeatures={Size=11}
]{Carlito-Regular.ttf} % Set Font
% Document

\begin{document}
% Title
\noindent
{\fontsize{40pt}{42pt}
    \selectfont \textbf{ML strengthened for next generation.}
}

\vspace{\baselineskip} % Prettier line

% Author + Date
\noindent % No Annoying Space at start of paragraph
\textbf{\emph{By Clemens Schütz}} % TextBF = Bold, Emph = Italic

\noindent % same here
\emph{Spaceport News}


\begin{multicols}{3}% Start layout

\noindent
\lettrine[lraise=0.1, nindent=0em, slope=-.5em]{\textbf{H}}{umanities} need
to explore that unbeknownst to it is as old as
the human thyself. 
\noindent
No matter what was discovered and uncovered, the human yearned for more - For better, or for worse. 
It comes to no suprise, that since the day that humanity has learned about what lies beyond
the blue tint of our sky, it has longed to
explore it.
And we have managed to do so well.
We have reconnoitred the hills of the moon
which seems to be so afar just 63 years ago - A Timespan that while seeming lingering is truly insignificant in the long history of human exploration.
After fulfilling this long-standing wish of humanity, it
would have been uncustomary, unnatural even, for humanity to be satisfied.
It set it's eyes on it's next target. Our Red Neighbour.
Yet still, 5 duodecennials after mankind has not been able to land a human being on our neighbour.
Even though it may seem as if little has progressed, this seems so only at first glance.
At second glance, one realizes just how much we have progressed.
If the quest to land a human on mars was a race, we would be approaching the finish line at exponential speed.
Still, a plethora of issues must still be adressed, questions must be answered and math must be solved. % Look at this again
\\ % Newline
One of said questions is the long-standing question of where to land.
Some consider this a question easily solved, but this also seems so only at first glance - Just like so many problems in the world of aerospace engineering.
If a group of 100 scientists, each of them in different fields, was asked the question asked above, responses would diverge from each other massively. 
This is why the will of scientists must not influence the ultimate landing spot.
Additionally, space agencies should first contrive to land a human on mars before making promises to researchers.
The above reasons contribute to the idea that the ideal landing spot should be decided through data.
The data used in this article are very simple maps.

\end{multicols}
\end{document}